%%%%%%%%%%%%%%%%%%%%%%%%%%%%%%%%%%%%%%%%%
% Beamer Presentation
% LaTeX Template
% Version 1.0 (10/11/12)
%
% This template has been downloaded from:
% http://www.LaTeXTemplates.com
%
% License:
% CC BY-NC-SA 3.0 (http://creativecommons.org/licenses/by-nc-sa/3.0/)
%
%%%%%%%%%%%%%%%%%%%%%%%%%%%%%%%%%%%%%%%%%

%----------------------------------------------------------------------------------------
%   PACKAGES AND THEMES
%----------------------------------------------------------------------------------------

\documentclass{beamer}

\mode<presentation> {

\usetheme{Madrid}
\usecolortheme{whale}
}
\usepackage[utf8]{inputenc}
\usepackage{listings}
\usepackage{graphicx}
\usepackage{booktabs}
\usepackage{color}

%----------------------------------------------------------------------------------------
%   TITLE PAGE
%----------------------------------------------------------------------------------------

\title[Secure Coding]{Secure Coding }

\author{Martin Vesterlund}
\institute[Cybercom]
{
Cybercom Group AB \\
\medskip
\textit{martin.vesterlund@cybercom.com}
}
\date{\today}

\begin{document}
\lstset{
  language=Java,
  showspaces=false,
  showstringspaces=false,
  keywordstyle=\color{blue},
  stringstyle=\color{red}
}
\begin{frame}
\titlepage % Print the title page as the first slide
\end{frame}

\begin{frame}
\frametitle{Overview}
\tableofcontents
\end{frame}

%----------------------------------------------------------------------------------------
%   PRESENTATION SLIDES
%----------------------------------------------------------------------------------------
\section{whoami}
\begin{frame}
\frametitle{whoami}
\begin{itemize}
  \item Former BTH student - Master of science in Computer Security
  \item Consultant at Cybercom Karlskrona
  \begin{itemize}
    \item Frontend developer at Försäkringskassan
    \item Frontend developer at Ericsson
    \item Backend developer at Ericsson (focused on integrationtests)
    \item Development Support/QA at Telenor
    \begin{itemize}
      \item Test automation
      \item Secure development
      \item System administrator build environment
      \item ...and a lot of other things
    \end{itemize}
  \end{itemize}
  \item Certified Agile Test Driven Development
\end{itemize}
\end{frame}
%------------------------------------------------
\section{Foundation}
%------------------------------------------------

\subsection{Code Standard}
\begin{frame}
\frametitle{Code Standard}
\begin{itemize}
  \item Set of guidelines, or rules, about how the code shall be written
  \item General formatting rules, such as indentation style, line length and such
  \item Naming of variables, constants, functions, classes ...
  \item How to organize the code on various levels
  \item Can contain specific guidelines of which patterns to follow
\end{itemize}
\end{frame}
\begin{frame}
\frametitle{Code Standard - From a security perspective}
\begin{itemize}
  \item Increase the overall readability of the code
  \item Reduces complexity (if specified in the standard)
  \item Can be used to enforce use of specific patterns
  \item Can be used to prohibit use of certain dangerous constructs
\end{itemize}
\end{frame}
\begin{frame}

\subsection{Code Review}
\begin{frame}
\frametitle{Code Review}
\begin{itemize}
  \item Manual and Automatic
  \item A second opionon on the changed code before it goes live
  \item Make sure the changed code follow the coding standard
  \item Spread information about what has been done in the code
  \item Make sure the change is applicable and called for
  \item Find bugs and problematic/unclear code
\end{itemize}
\end{frame}

\subsection{Static analysis and tests}
\begin{frame}
\frametitle{Static analysis}
\begin{itemize}
  \item Should be part of automated code review
  \item Does not test the function of the code (or if can be compiled)
  \item Used mainly for checking compliance with code standard
  \item Can be used to highlight some problematic code constructs, such as
  \begin{itemize}
    \item Calls to unsafe functions
    \item SQL-statements with undbound variables
    \item Possible use of unsafe input
    \item Might find potential logical bugs
  \end{itemize}
\end{itemize}
\end{frame}

\begin{frame}
\frametitle{Tests}
\begin{itemize}
  \item Should be part of automated code review/build
  \item Does test the function of the code
  \item Can be seen as documentation of the code
  \item Should be mappable against requirements and/or user stories
  \item There is no difference between normal functional/behavioural tests and security tests!
\end{itemize}
\end{frame}

\section{Secure Code - Crash course}

\subsection{What defines secure code?}
\begin{frame}
\frametitle{What defines secure code?}
\begin{itemize}
  \item Overall low complexity
  \item Testable
  \item High readability (e.g. small classes, descriptive names, shallow nesting)
  \item Handles errors gracefully (expects the unexpected)
  \item Does not build upon assumptions about the surroundings (i.e. external interfaces)
  \item Defines and enforce trust boundaries
\end{itemize}
\end{frame}

\subsection{Requirements, Tasks, User Stories and Threat modeling}
\begin{frame}
\frametitle{Requirements, Tasks and User Stories}
\begin{itemize}
\item
\end{itemize}
\end{frame}

\subsection{Secure code}
\begin{frame}
\frametitle{Requirements, Tasks and User Stories}
\begin{itemize}
\item
\end{itemize}
\end{frame}

\subsection{Security tests}
\begin{frame}
\frametitle{Requirements, Tasks and User Stories}
\begin{itemize}
\item
\end{itemize}
\end{frame}

\subsection{Security code review}
\begin{frame}
\frametitle{Requirements, Tasks and User Stories}
\begin{itemize}
\item
\end{itemize}
\end{frame}

\iffalse
\begin{frame}
\frametitle{Blocks of Highlighted Text}
\begin{block}{Block 1}
Lorem ipsum dolor sit amet, consectetur adipiscing elit. Integer lectus nisl, ultricies in feugiat rutrum, porttitor sit amet augue. Aliquam ut tortor mauris. Sed volutpat ante purus, quis accumsan dolor.
\end{block}

\begin{block}{Block 2}
Pellentesque sed tellus purus. Class aptent taciti sociosqu ad litora torquent per conubia nostra, per inceptos himenaeos. Vestibulum quis magna at risus dictum tempor eu vitae velit.
\end{block}

\begin{block}{Block 3}
Suspendisse tincidunt sagittis gravida. Curabitur condimentum, enim sed venenatis rutrum, ipsum neque consectetur orci, sed blandit justo nisi ac lacus.
\end{block}
\end{frame}

\begin{frame}[fragile]
\frametitle{Code}
\begin{example}[Code]
\begin{lstlisting}
@Mock
private Belly belly;
private CookieMonster cookieMonster = new CookieMonster(belly);

@Test
public void cookieMonsterHaveNoCookies() {
  assertThat(cookieMonster.haveCookies())
    .as("Should reply with no cookie but did not")
    .isEqualToIgnoringCase("Cookiemonster have no cookie");
}
\end{lstlisting}
\end{example}
\end{frame}

\begin{frame}[fragile] % Need to use the fragile option when verbatim is used in the slide
\frametitle{Citation}
An example of the \verb|\cite| command to cite within the presentation:\\~

This statement requires citation \cite{p1}.
\end{frame}

%------------------------------------------------

\begin{frame}
\frametitle{References}
\footnotesize{
\begin{thebibliography}{99} % Beamer does not support BibTeX so references must be inserted manually as below
\bibitem[Smith, 2012]{p1} John Smith (2012)
\newblock Title of the publication
\newblock \emph{Journal Name} 12(3), 45 -- 678.
\end{thebibliography}
}
\end{frame}
\fi
%------------------------------------------------

\begin{frame}
\Huge{\centerline{The End}}
\end{frame}

%----------------------------------------------------------------------------------------

\end{document}
