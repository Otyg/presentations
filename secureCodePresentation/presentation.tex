%%%%%%%%%%%%%%%%%%%%%%%%%%%%%%%%%%%%%%%%%
% Beamer Presentation
% LaTeX Template
% Version 1.0 (10/11/12)
%
% This template has been downloaded from:
% http://www.LaTeXTemplates.com
%
% License:
% CC BY-NC-SA 3.0 (http://creativecommons.org/licenses/by-nc-sa/3.0/)
%
%%%%%%%%%%%%%%%%%%%%%%%%%%%%%%%%%%%%%%%%%

%----------------------------------------------------------------------------------------
%   PACKAGES AND THEMES
%----------------------------------------------------------------------------------------

\documentclass{beamer}

\mode<presentation> {

\usetheme{Madrid}
\usecolortheme{whale}
}
\usepackage[utf8]{inputenc}
\usepackage{listings}
\usepackage{graphicx}
\usepackage{booktabs}
\usepackage{color}

%----------------------------------------------------------------------------------------
%   TITLE PAGE
%----------------------------------------------------------------------------------------

\title[Secure Coding]{Secure Coding }

\author{Martin Vesterlund}
\institute[Cybercom]
{
Cybercom Group AB \\
\medskip
\textit{martin.vesterlund@cybercom.com}
}
\date{\today}

\begin{document}
\lstset{
  language=Java,
  showspaces=false,
  showstringspaces=false,
  keywordstyle=\color{blue},
  stringstyle=\color{red}
}
\begin{frame}
\titlepage % Print the title page as the first slide
\end{frame}

\begin{frame}
\frametitle{Overview}
\tableofcontents
\end{frame}

%----------------------------------------------------------------------------------------
%   PRESENTATION SLIDES
%----------------------------------------------------------------------------------------
\section{whoami}
\begin{frame}
\frametitle{whoami}
\begin{itemize}
  \item Former BTH student - Master of science in Computer Security
  \item Consultant at Cybercom Karlskrona
  \begin{itemize}
    \item Frontend developer at Försäkringskassan
    \item Frontend developer at Ericsson
    \item Backend developer at Ericsson (focused on integrationtests)
    \item Development Support/QA at Telenor
    \begin{itemize}
      \item Test automation
      \item Secure development
      \item System administrator build environment
      \item ...and a lot of other things
    \end{itemize}
  \end{itemize}
  \item Certified Agile Test Driven Development
\end{itemize}
\end{frame}
%------------------------------------------------
\section{Foundation}
%------------------------------------------------

\subsection{Code Standard}
\begin{frame}
\frametitle{Code Standard}
\begin{itemize}
\item
\end{itemize}
\end{frame}

\subsection{Requirements and Tasks}
\begin{frame}
\frametitle{Requirements and Tasks}
\begin{itemize}
\item
\end{itemize}
\end{frame}

\subsection{Code Review}
\begin{frame}
\frametitle{Code Reivew}
\begin{itemize}
\item
\end{itemize}
\end{frame}

\subsection{Tests}
\begin{frame}
\frametitle{Tests}
\begin{itemize}
\item
\end{itemize}
\end{frame}

\begin{frame}
\frametitle{Blocks of Highlighted Text}
\begin{block}{Block 1}
Lorem ipsum dolor sit amet, consectetur adipiscing elit. Integer lectus nisl, ultricies in feugiat rutrum, porttitor sit amet augue. Aliquam ut tortor mauris. Sed volutpat ante purus, quis accumsan dolor.
\end{block}

\begin{block}{Block 2}
Pellentesque sed tellus purus. Class aptent taciti sociosqu ad litora torquent per conubia nostra, per inceptos himenaeos. Vestibulum quis magna at risus dictum tempor eu vitae velit.
\end{block}

\begin{block}{Block 3}
Suspendisse tincidunt sagittis gravida. Curabitur condimentum, enim sed venenatis rutrum, ipsum neque consectetur orci, sed blandit justo nisi ac lacus.
\end{block}
\end{frame}

\begin{frame}[fragile]
\frametitle{Code}
\begin{example}[Code]
\begin{lstlisting}
@Mock
private Belly belly;
private CookieMonster cookieMonster = new CookieMonster(belly);

@Test
public void cookieMonsterHaveNoCookies() {
  assertThat(cookieMonster.haveCookies())
    .as("Should reply with no cookie but did not")
    .isEqualToIgnoringCase("Cookiemonster have no cookie");
}
\end{lstlisting}
\end{example}
\end{frame}

\begin{frame}[fragile] % Need to use the fragile option when verbatim is used in the slide
\frametitle{Citation}
An example of the \verb|\cite| command to cite within the presentation:\\~

This statement requires citation \cite{p1}.
\end{frame}

%------------------------------------------------

\begin{frame}
\frametitle{References}
\footnotesize{
\begin{thebibliography}{99} % Beamer does not support BibTeX so references must be inserted manually as below
\bibitem[Smith, 2012]{p1} John Smith (2012)
\newblock Title of the publication
\newblock \emph{Journal Name} 12(3), 45 -- 678.
\end{thebibliography}
}
\end{frame}

%------------------------------------------------

\begin{frame}
\Huge{\centerline{The End}}
\end{frame}

%----------------------------------------------------------------------------------------

\end{document}
